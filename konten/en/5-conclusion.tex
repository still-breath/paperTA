% Ubah judul dan label berikut sesuai dengan yang diinginkan.
\section{Conclusion}
\label{sec:conclusion}

% Ubah paragraf-paragraf pada bagian ini sesuai dengan yang diinginkan.

This study developed a vehicle speed estimation system using a DJI Phantom 4 drone and YOLOv8 deep learning model, integrated with a Jetson Nano for edge processing. The system demonstrated its ability to detect vehicles and estimate their speed with reasonable accuracy in both daytime and nighttime conditions.

The results show that the system's accuracy in vehicle detection and speed estimation is affected by environmental factors, such as lighting conditions and altitude. During daytime, the model achieved a high mAP50 of 0.92997 at epoch 72, while nighttime testing showed slightly lower performance but still promising results. Speed estimation errors ranged from 10\% to 16.8\%, depending on the test conditions, with higher errors observed at higher speeds and greater altitudes.

The system's performance is limited by the hardware capabilities of the Jetson Nano, with a frame rate of 15 FPS and some latency in real-time processing. However, the system remains functional for low-speed vehicle detection and can be further optimized for higher accuracy and performance with improved hardware and software optimizations.

In conclusion, the proposed system is a promising solution for real-time traffic monitoring and speed estimation, especially in controlled environments. Future improvements, including hardware upgrades and further model fine-tuning, can enhance its accuracy and applicability in various real-world scenarios.
