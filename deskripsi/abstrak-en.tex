% Mengubah keterangan `Abstract` ke bahasa indonesia.
% Hapus bagian ini untuk mengembalikan ke format awal.
% \renewcommand\abstractname{Abstrak}

\begin{abstract}

  % Ubah paragraf berikut sesuai dengan abstrak dari penelitian.
The increase in population in Indonesia is directly followed by an increase in the risk
of traffic congestion and accidents. From the data obtained, it can be seen that from year to
year the number of motorized vehicles in Indonesia is increasing, followed by the number of
traffic accidents that occur, where one of the factors is speed limit violations. To overcome
this problem, an aerial monitoring tool was developed using drones that can calculate vehicle
speed estimates supported by video image processing technology and computing programs in
their calculations. Drones are used because of their ability to monitor directly from the air
with remote control. This system is also developed based on Jetson Nano so that it can be run
without the need for an internet network. The method used in processing this video image is to
utilize YOLOv8 for accurate and precise results.


\end{abstract}

% Mengubah keterangan `Index terms` ke bahasa indonesia.
% Hapus bagian ini untuk mengembalikan ke format awal.
% \renewcommand\IEEEkeywordsname{Kata kunci}

\begin{IEEEkeywords}

  % Ubah kata-kata berikut sesuai dengan kata kunci dari penelitian.
  Drone, traffic accident, speed estimation, jetson nano, YOLOv8

\end{IEEEkeywords}
